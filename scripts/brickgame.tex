\documentclass[12pt]{article}
\usepackage[utf8]{inputenc}
\usepackage[T1]{fontenc}
\usepackage{graphicx}
\usepackage{fancyvrb}
\usepackage{hyperref}
\usepackage[margin=1in]{geometry}
\usepackage{listings}
\usepackage{xcolor}
\usepackage{tabularx}
\usepackage{float}
\usepackage{tcolorbox}

% Define colors for code listings
\definecolor{codegreen}{rgb}{0,0.6,0}
\definecolor{codegray}{rgb}{0.5,0.5,0.5}
\definecolor{codepurple}{rgb}{0.58,0,0.82}
\definecolor{backcolour}{rgb}{0.95,0.95,0.92}

\lstdefinestyle{mystyle}{
    backgroundcolor=\color{backcolour},   
    commentstyle=\color{codegreen},
    keywordstyle=\color{magenta},
    numberstyle=\tiny\color{codegray},
    stringstyle=\color{codepurple},
    basicstyle=\ttfamily\footnotesize,
    breakatwhitespace=false,         
    breaklines=true,                 
    captionpos=b,                    
    keepspaces=true,                 
    numbers=left,                    
    numbersep=5pt,                  
    showspaces=false,                
    showstringspaces=false,
    showtabs=false,                  
    tabsize=2,
    frame=single
}

\lstset{style=mystyle}

% Custom boxes for notes and warnings
\newtcolorbox{notebox}{
    colback=blue!5!white,
    colframe=blue!75!black,
    title=Note,
    sharp corners
}

\newtcolorbox{warningbox}{
    colback=red!5!white,
    colframe=red!75!black,
    title=Warning,
    sharp corners
}

\title{BrickGame Documentation}
\author{BrickGame Development Team}
\date{Version 1.0 \\ \today}

\begin{document}

\maketitle

\begin{abstract}
This document provides comprehensive documentation for the BrickGame project, including installation instructions, build procedures, project structure, dependency management, and usage guidelines. BrickGame is a classic brick game implementation featuring both terminal-based (CLI) and desktop graphical interfaces.
\end{abstract}

\tableofcontents
\newpage

\section{Introduction}
BrickGame is a modern implementation of classic brick games built using C++20. The project features modular architecture with separate game libraries and multiple user interfaces including Qt-based desktop GUI and ncurses-based CLI.

\section{System Requirements}

\subsection{Prerequisites}
\begin{itemize}
    \item \textbf{CMake} 3.16 or higher
    \item \textbf{C++ Compiler} with C++20 support (GCC 10+, Clang 10+, or MSVC 2019+)
    \item \textbf{Build System}: Make or Ninja
    \item \textbf{Linux Distribution}: Ubuntu/Debian, Fedora/RHEL, or Arch Linux
\end{itemize}

\section{Dependencies}

\subsection{Core Dependencies}

\begin{tabularx}{\textwidth}{|l|X|}
\hline
\textbf{Dependency} & \textbf{Purpose} \\
\hline
GTest & Google Test framework for unit testing \\
\hline
Check & Check unit testing framework (alternative) \\
\hline
Subunit & Test reporting protocol \\
\hline
Ncurses & Terminal UI library for CLI interface \\
\hline
Menu & Ncurses extension for menu handling \\
\hline
Qt5/Qt6 & GUI framework for desktop interface \\
\hline
SFML & Simple and Fast Multimedia Library (if used) \\
\hline
X11 & X Window System development libraries \\
\hline
\end{tabularx}

\subsection{Installation by Linux Distribution}

\subsubsection{Ubuntu/Debian}
\begin{lstlisting}[language=bash]
# Core build tools
sudo apt update
sudo apt install build-essential cmake ninja-build pkg-config

# Testing frameworks
sudo apt install libgtest-dev libgmock-dev googletest
sudo apt install check libsubunit-dev

# CLI dependencies (ncurses)
sudo apt install libncurses5-dev libncursesw5-dev libmenu-cache-dev

# Qt dependencies
sudo apt install qt6-base-dev qt6-tools-dev-tools  # For Qt6
# OR for Qt5:
sudo apt install qt5-default qtbase5-dev qttools5-dev-tools

# Optional: SFML for alternative GUI
sudo apt install libsfml-dev

# X11 development libraries
sudo apt install libx11-dev libxrandr-dev libxext-dev
\end{lstlisting}

\subsubsection{Fedora/RHEL/CentOS}
\begin{lstlisting}[language=bash]
# Core build tools
sudo dnf groupinstall "Development Tools"
sudo dnf install cmake ninja-build pkgconfig

# Testing frameworks
sudo dnf install gtest-devel gmock-devel
sudo dnf install check-devel subunit-devel

# CLI dependencies
sudo dnf install ncurses-devel

# Qt dependencies
sudo dnf install qt6-qtbase-devel qt6-tools-devel  # For Qt6
# OR for Qt5:
sudo dnf install qt5-qtbase-devel qt5-tools-devel

# Optional: SFML
sudo dnf install SFML-devel

# X11 libraries (usually included in development tools)
sudo dnf install libX11-devel libXrandr-devel libXext-devel
\end{lstlisting}

\subsubsection{Arch Linux/Manjaro}
\begin{lstlisting}[language=bash]
# Core build tools
sudo pacman -S base-devel cmake ninja pkg-config

# Testing frameworks
sudo pacman -S gtest gmock
sudo pacman -S check subunit

# CLI dependencies
sudo pacman -S ncurses

# Qt dependencies
sudo pacman -S qt6-base qt6-tools  # For Qt6
# OR for Qt5:
sudo pacman -S qt5-base qt5-tools

# Optional: SFML
sudo pacman -S sfml

# X11 libraries (usually installed with base-devel)
sudo pacman -S libx11 libxrandr libxext
\end{lstlisting}

\subsection{Dependency Troubleshooting}

\subsubsection{Common Dependency Issues}

\begin{warningbox}
\textbf{GTest Not Found}: If CMake cannot find GTest, you may need to build it from source:
\end{warningbox}

\begin{lstlisting}[language=bash]
# On Ubuntu/Debian if GTest is not properly installed
sudo apt install libgtest-dev
cd /usr/src/gtest
sudo cmake CMakeLists.txt
sudo make
sudo cp lib/*.a /usr/lib
\end{lstlisting}

\begin{warningbox}
\textbf{Qt Not Found}: Ensure the correct Qt version is installed and QTDIR is set:
\end{warningbox}

\begin{lstlisting}[language=bash]
# Check available Qt versions
qtchooser -list-versions

# Set default Qt version (on Ubuntu)
sudo update-alternatives --config qtchooser

# Verify Qt installation
qmake --version
\end{lstlisting}

\begin{warningbox}
\textbf{Ncurses Development Files Missing}: Install both runtime and development packages:
\end{warningbox}

\begin{lstlisting}[language=bash]
# Ubuntu/Debian
sudo apt install libncurses5-dev libncursesw5-dev

# Fedora/RHEL
sudo dnf install ncurses-devel

# Arch Linux
sudo pacman -S ncurses
\end{lstlisting}

\subsubsection{Verifying Dependencies}
\begin{lstlisting}[language=bash]
# Check if dependencies are available
pkg-config --list-all | grep -E "(Qt|ncurses|check|gtest)"

# Verify compiler can find headers
echo '#include <ncurses.h>' | gcc -E - >/dev/null
echo '#include <gtest/gtest.h>' | g++ -E - >/dev/null
echo '#include <check.h>' | gcc -E - >/dev/null
\end{lstlisting}

\section{Installation}

\subsection{Obtaining the Source Code}
\begin{lstlisting}[language=bash]
# Clone the repository
git clone <repository-url>
cd BrickGame
\end{lstlisting}

\subsection{Building from Source}

\subsubsection{Standard Build Process}
\begin{lstlisting}[language=bash]
# Build release version (recommended for end users)
make release

# Build debug version with sanitizers
make debug

# Build with tests
make test
\end{lstlisting}

\subsubsection{Advanced Build Options}
\begin{lstlisting}[language=bash]
# Configure with custom options
cmake -B build -DBUILD_DESKTOP_GUI=ON -DBUILD_CLI_GUI=ON -DBUILD_TESTS=OFF

# Build using Ninja for faster compilation
cmake -B build -G Ninja
ninja -C build

# Build with specific Qt version
cmake -B build -DQT_VERSION=6
\end{lstlisting}

\subsection{Installation Methods}

\subsubsection{User Installation}
\begin{lstlisting}[language=bash]
# Install to user directory (~/.local/share/BrickGame)
make install

# Install to custom location
make install INSTALL_PREFIX=/path/to/custom/location
\end{lstlisting}

\subsubsection{System-wide Installation}
\begin{lstlisting}[language=bash]
# Install system-wide (requires privileges)
make system-install

# Or manually with CMake
sudo cmake --install build --prefix /usr/local
\end{lstlisting}

\subsection{Verifying Installation}
After installation, verify the installation by running:
\begin{lstlisting}[language=bash]
# Test CLI version
BrickGameCLI --version

# Test desktop version  
BrickGameDesktop --version
\end{lstlisting}

\section{Project Structure}

\subsection{Source Code Organization}
\begin{lstlisting}[basicstyle=\ttfamily\small]
BrickGame/
|-- src/
|   |-- brick_game/
|   |   |-- snake/          # Snake game library
|   |   |-- tetris/         # Tetris game library
|   |-- gui/
|   |   |-- cli/           # Command-line interface (ncurses)
|   |   |-- desktop/       # Graphical interface (Qt)
|   |-- tests/             # Test suites (GTest/Check)
|-- include/               # Header files
|-- materials/            # Game assets and resources
|-- scripts/              # Build and installation scripts
|-- docs/                 # Documentation
\end{lstlisting}

\subsection{Build Artifacts}
\begin{itemize}
    \item \texttt{build/Release/} - Optimized release binaries
    \item \texttt{build/Debug/} - Debug binaries with symbols
    \item \texttt{build/lib/} - Compiled libraries (\texttt{libsnake.a}, \texttt{libtetris.a})
    \item \texttt{build/bin/} - Executable binaries
    \item \texttt{build/tests/} - Test executables
\end{itemize}

\section{Usage}

\subsection{Command Line Interface (ncurses)}
Run the terminal-based version:
\begin{lstlisting}[language=bash]
./build/Release/bin/BrickGameCLI [options]
\end{lstlisting}

\textbf{Available Options:}
\begin{itemize}
    \item \texttt{--game <snake|tetris>} - Select game (default: tetris)
    \item \texttt{--help} - Show help message
    \item \texttt{--version} - Display version information
    \item \texttt{--level <n>} - Start at specific level
\end{itemize}

\subsection{Desktop Graphical Interface (Qt)}
Launch the graphical version:
\begin{lstlisting}[language=bash]
./build/Release/bin/BrickGameDesktop
\end{lstlisting}

Or use the desktop application menu after installation.

\subsection{Game Controls}

\subsubsection{Snake Game}
\begin{itemize}
    \item \textbf{Arrow Keys} - Change snake direction
    \item \textbf{Space} - Pause/resume game
    \item \textbf{ESC} - Exit to main menu
    \item \textbf{Q} - Quit game
\end{itemize}

\subsubsection{Tetris Game}
\begin{itemize}
    \item \textbf{Arrow Left/Right} - Move piece horizontally
    \item \textbf{Arrow Down} - Soft drop
    \item \textbf{Arrow Up} - Rotate piece
    \item \textbf{Space} - Hard drop
    \item \textbf{ESC} - Pause/exit
    \item \textbf{P} - Pause game
\end{itemize}

\section{Testing}

\subsection{Running Test Suite}
\begin{lstlisting}[language=bash]
# Build and run all tests
make test

# Run specific test suites
./build/Release/tests/snake/snake_tests
./build/Release/tests/tetris/tetris_tests

# Run tests with verbose output
ctest -V
\end{lstlisting}

\subsection{Code Coverage}
Generate coverage reports (requires lcov):
\begin{lstlisting}[language=bash]
# Install coverage tools (Ubuntu/Debian)
sudo apt install lcov genhtml

# Generate HTML coverage report
make coverage

# View coverage report
xdg-open build/coverage_report/index.html
\end{lstlisting}

\section{Development}

\subsection{Build Configuration Options}
The project supports several CMake options:

\begin{tabularx}{\textwidth}{|l|X|}
\hline
\textbf{Option} & \textbf{Description} \\
\hline
\texttt{BUILD\_DESKTOP\_GUI} & Build Qt graphical interface (default: ON) \\
\hline
\texttt{BUILD\_CLI\_GUI} & Build ncurses command-line interface (default: ON) \\
\hline
\texttt{BUILD\_TESTS} & Build test suites with GTest/Check (default: OFF) \\
\hline
\texttt{BUILD\_SNAKE\_LIB} & Build Snake game library (default: ON) \\
\hline
\texttt{BUILD\_TETRIS\_LIB} & Build Tetris game library (default: ON) \\
\hline
\texttt{ENABLE\_SANITIZER} & Enable AddressSanitizer (default: OFF) \\
\hline
\texttt{QT\_VERSION} & Specify Qt version (5 or 6) \\
\hline
\end{tabularx}

\subsection{Makefile Targets}

\begin{tabularx}{\textwidth}{|l|X|}
\hline
\textbf{Target} & \textbf{Description} \\
\hline
\texttt{all} & Build project (default target) \\
\hline
\texttt{release} & Build optimized release version \\
\hline
\texttt{debug} & Build debug version with sanitizers \\
\hline
\texttt{test} & Build and run all tests \\
\hline
\texttt{coverage} & Generate code coverage report \\
\hline
\texttt{install} & Install to user directory \\
\hline
\texttt{system-install} & Install system-wide \\
\hline
\texttt{uninstall} & Remove installed files \\
\hline
\texttt{clean} & Remove build directory \\
\hline
\texttt{dvi} & Generate documentation \\
\hline
\texttt{dist} & Create distribution tarball \\
\hline
\texttt{help} & Show available targets \\
\hline
\end{tabularx}

\section{Troubleshooting}

\subsection{Common Build Issues}

\subsubsection{CMake Configuration Fails}
\begin{itemize}
\item Ensure CMake version is 3.16 or higher: \verb|cmake --version|
\item Check that C++20 compatible compiler is installed: \verb|g++ --version|
\item Verify all dependencies are installed (see Section 3)
\item Clear CMake cache: \verb|rm -rf build && mkdir build|
\end{itemize}

\subsubsection{Missing Dependencies Errors}
\begin{lstlisting}[language=bash]
# If you get "Could NOT find Qt5" or similar errors:
sudo apt install qtbase5-dev  # For Ubuntu/Debian
sudo dnf install qt5-qtbase-devel  # For Fedora

# If ncurses is missing:
sudo apt install libncurses5-dev  # Ubuntu/Debian
sudo dnf install ncurses-devel    # Fedora

# If GTest is missing:
sudo apt install libgtest-dev     # Ubuntu/Debian
sudo dnf install gtest-devel      # Fedora
\end{lstlisting}

\subsubsection{Linker Errors}
\begin{itemize}
\item Ensure development packages are installed (packages ending in \texttt{-dev} or \texttt{-devel})
\item Check that library paths are correct: \verb|ldconfig -p | grep ncurses|
\item Verify pkg-config can find packages: \verb|pkg-config --cflags --libs ncurses|
\end{itemize}

\subsubsection{Runtime Issues}
\begin{itemize}
\item \textbf{CLI version not starting}: Check terminal compatibility and ncurses installation
\item \textbf{Qt version not found}: Set \texttt{QT\_SELECT=5} or \texttt{QT\_SELECT=6} environment variable
\item \textbf{Permission denied}: Ensure binaries have execute permissions: \verb|chmod +x build/Release/bin/*|
\end{itemize}

\subsection{Debugging Tips}
\begin{lstlisting}[language=bash]
# Build with debug symbols
make debug

# Run with gdb
gdb ./build/Debug/bin/BrickGameCLI

# Check shared libraries
ldd ./build/Release/bin/BrickGameDesktop

# Verbose CMake output
cmake -B build -DCMAKE_VERBOSE_MAKEFILE=ON
\end{lstlisting}

\section{License}
This project is distributed under the terms of the LICENSE file included in the source distribution. See the LICENSE file for complete terms and conditions.

\section{Support}
For issues, questions, or contributions:
\begin{itemize}
    \item Check the project documentation and this file
    \item Review existing issues in the repository
    \item Create a new issue with detailed description including:
    \begin{itemize}
        \item Linux distribution and version
        \item Installed dependencies and versions
        \item Complete error messages
        \item Steps to reproduce the issue
    \end{itemize}
\end{itemize}

\end{document}